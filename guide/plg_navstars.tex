% Part of Stellarium User Guide 0.20+
% History:
% 2020-02-08 Rewritten chapter.

\newpage
\section{Navigational Stars Plugin}
\label{sec:plugins:NavigationalStars}
\sectionauthor*{Andy Kirkham}
\begin{figure}[ht]
\includegraphics[width=\textwidth]{navstars.jpg}
\caption{Navigational stars on the screen}
\label{fig:plugin:NavigationalStars}
\end{figure}

\subsection{Introduction}
\label{sec:plugin:NavigationalStars:Introduction}

Modern times have witnessed rapid growth in technological devices 
that enable people to navigate the natural world in remarkable ways. 
The most popular among these is the Global Positioning System 
(GPS)\footnote{\url{https://en.wikipedia.org/wiki/Global_Positioning_System}} 
that, at the touch of a button, can reveal your location anywhere on the 
surface of the Earth along with an accurate time piece.

However, despite the modern marvels, there are times when technology can malfunction. 
Poor signal reception, damaged devices or simply dead batteries. When it's critical, 
relying entirerly on these devices can be a poor decision. It's for this reason, that 
still today, The International Maritime Organisation still requires commercial vessels 
to carry an appropiate Nautical Almanac onboard and requires trained naval officers to
be able to make use of them.

\subsection{History}
\label{sec:plugin:NavigationalStars:History}




\subsection{Configuarion}
\label{sec:plugin:NavigationalStars:Configuarion}


This plugin marks navigational stars from a selected set:

\begin{description}
	\item[Anglo-American] --- the 57 "selected stars" that are listed in \emph{The Nautical Almanac}\footnote{The Nautical Almanac
		website -- \url{http://aa.usno.navy.mil/publications/docs/na.php}} jointly published by Her Majesty's Nautical Almanac Office and the US Naval Observatory since 1958; consequently, these stars are also used in navigational aids such as the \emph{2102-D Star Finder}\footnote{Rude Starfinder 2102-D
		description and usage instruction --
		\url{http://oceannavigation.blogspot.ru/2008/12/rude-starfinder-2102-d.html}} and \emph{Identifier}. 
	\item[French] --- the 81 stars that are listed in the \emph{Ephémérides Nautiques} published by the French Bureau des Longitudes.
	\item[Russian] --- the 160 stars that are listed in the Russian Nautical Almanac.
\end{description}
If enabled (see section~\ref{sec:Plugins:EnablingPlugins}), just click
on the Sextant button \guibutton{0.6}{bt_NavStars-off.png} on
the bottom toolbar to display markers for the navigational stars. This
can help you in training your skills in astronomical navigation before
you cruise the ocean in the traditional way, with your sextant and
chronometer.


\subsection{Section \big[NavigationalStars\big] in config.ini file}
%\label{sec:plugins:NavigationalStars:configuration}

You can edit \file{config.ini} file by yourself for changes of the
settings for the Navigational Stars plugin -- just make it carefully!

\noindent%
\begin{tabularx}{\textwidth}{l|l|X}\toprule
\emph{ID}			& \emph{Type} 	& \emph{Description}\\\midrule
navstars\_color 	& R,G,B 		& Color of markers of navigational stars  \\
enable\_at\_startup & bool 		    & Set to \emph{true} to display navigational stars at startup of planetarium  \\
current\_ns\_set	& string		& Current set of navigational stars. Possible values: \emph{AngloAmerican}, \emph{French} and \emph{Russian}. \\
\bottomrule
\end{tabularx}

